\section{Data}
% \textit{The Data section (indicative length : 1 page) should provide a deep analysis of the data used for the experiment. In particular, we are interested here in your capacity to provide relevant and thoughtful feedbacks on the data and to demonstrate that you master some "data diagnosis" tools that have been dealt with in the lectures/tutorials.}

\subsection{BOLD5000 dataset}

We used the BOLD5000 database \cite{chang_bold5000_2019} in our experiments. It is a large-scale fMRI dataset that captures brain scans from four patients as they view over 5,000 images. The dataset covers a wide range of visual features, categories, and semantics, and can be used to test various hypotheses and models related to visual cognition. \\
fMRI produces 4D images, relying on the principle that localized neural activity induces changes in metabolism and blood flow.
% \red{, resulting in fluctuations in the concentrations of oxyhemoglobin\footnote{Oxyhaemoglobin are the red cells in the blood that carry oxygen} and deoxyhemoglobin\footnote{the same red cells after they have delivered the oxygen}. Oxyhaemoglobin and deoxyhaemoglobin have different magnetic properties\footnote{Oxyhemoglobin is diamagnetic, while deoxyhemoglobin is paramagnetic}, and they affect the local magnetic field in different ways.}
To record cerebral activity during functional sessions, the scanner is tuned to detect this "Blood Oxygen Level Dependent" (BOLD) signal. 
Brain activity is measured in sessions that span several minutes, during which participants are presented with a variety of images. Simultaneously, participants engage in a valence judgment task for each stimulus, expressing their preference using the descriptors "like", "neutral", and "dislike", which are encoded with labels 1, 2, and 3.\\
As the mapping between descriptors and labels was not provided, we inferred it ourselves.

\subsection{Time series extraction and diagnosis}

For each region of interests (ROIs) in the dataset, we extracted the time series, which represents the temporal activity in the brain. We used Nilearn -a Python library that provides tools for neuroimaging data analysis- to extract the signals.
Signal extraction is usually achieved by averaging the fMRI time series across the voxels in a region \cite{varoquaux_learning_2013}. 

In this study, time series were $z$-scored, \ie shifted to zero mean and scaled to unit variance. We also took into account the confounds in the extraction, as there were provided in the database. Confounds are variables that can affect the brain signal and are not of interest in the study. They can include head motion, physiological noise, and scanner artefacts. In order to extract clean signals from brain regions, it is important to remove the effects of confounds from the data. 
% This is done by regressing out the confounds from the data before extracting the signals. 

The time series obtained from the regions of interest (ROIs) do not appear to be noisy, as shown in Figure \ref{fig:plot_TS}. One can assume that signal are weak stationary as their mean and autocorrelation does not appear to vary over time. The spectrogram plotted in figure \ref{spectrogram_TS} shows that their spectral properties does noes not change over time. They all exhibit the same variability and mean as they were z-scored. Their values range between $-2.5$ and $2.5$, except for some regions that have extreme values or outliers (see Fig \ref{fig:boxplot_TS}). They seem to have the same spectral properties except for the retrosplenial complex (RSC) region in left and right hemisphere which has a more chaotic spectrum. Those results are shown in figure \ref{fig:spectral_TS}.

\subsection{Brain Graph Construction}

In order to represent the strength of axonal connection (structural connectivity), authors of \cite{huang_graph_2018} defined $A_{ij}$ as a simple count of the number of streamlines - \ie estimated individual fibers that connect the regions - using diffusion spectrum imaging. However, since we did not have access to this information, we defined the adjacency matrix $\A$ in another manner. In our setting, it corresponds to the connectivity matrix between brain regions (functional connectivity). Indeed, the extracted signals can be used to compute a correlation matrix between the regions \cite{varoquaux_learning_2013}. It is a common metric for computing the edges between the nodes, and is depicted in figure \ref{fig:connectivity}.


