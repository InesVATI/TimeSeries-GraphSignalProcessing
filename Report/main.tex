\documentclass[11pt]{article}
\usepackage{theme}
\usepackage{shortcuts}
\usepackage[
style=phys,
]
{biblatex}
\addbibresource{ref.bib}

% Document parameters
% Document title
\title{Mini-Project (ML for Time Series) - MVA 2023/2024}
\author{
Mathis Reymond \email{mathis.reymond74@gmail.com} \\ % student 1
Inès Vati \email{ines.vati@eleves.enpc.fr} % student 2
}

\begin{document}
\maketitle

\paragraph{What is expected for these mini-projects?}
The goal of the exercise is to read (and understand) a research article, implement it (or find an implementation), test it on real data and comment on the results obtained.
Depending on the articles, the task will not always be the same: some articles are more theoretical or complex, others are in the direct line of the course, etc... It is therefore important to balance the exercise according to the article. For example, if you have reused an existing implementation, it is obvious that you will have to develop in a more detailed way the analysis of the results, the influence of the parameters etc... Do not hesitate to contact us by email if you wish to be guided.

\paragraph{The report}
 The report must be at most FIVE pages and use this template (excluding references). If needed, additional images and tables can be put in Appendix, but must be discussed in the main document. The report must contain a precise description of the work done, a description of the method, and the results of your tests. Please do not include source code! The report must clearly show the elements that you have done yourself and those that you have reused only, as well as the distribution of tasks within the team (see detailed plan below.)
 
 \paragraph{The source code}
In addition to this report, you will have to send us a Python notebook allowing to launch the code and to test it on data. For the data, you can find it on standard sites like Kaggle, or the site https://timeseriesclassification.com/ which contains a lot of signals!


\paragraph{The oral presentations}
They will last 10 minutes followed by 5 minutes of questions. The plan of the defense is the same as the one of the report: presentation of the work done, description of the method and analysis of the results.


\paragraph{Deadlines}
Two sessions will be available :
\begin{itemize}
 \item \textbf{Session 1}
 \begin{itemize}
  \item Deadline for report: December 18th (23:59)
  \item Oral presentations: December 20th and 22th (precise times TBA)
 \end{itemize}
\end{itemize}

\section{Introduction and contributions}

The Introduction section (indicative length : less than 1 page) should detail the scientific context of the article you chose, as well as the task that you want to solve (especially if you apply it on novel data). \textbf{The last paragraph of the introduction must contain the following information}:
\begin{itemize}
    \item Repartition of work between the two students
    \item Use of available source code or not, percentage of the source code that has been reused, etc.
    \item Use of existing experiments or new experiments (e.g. test of the influence of parameter that was not conducted in the original article, application of the method on a novel task/data set etc.)
    \item Improvement on the original method (e.g. new pre/post processing steps, grid search for optimal parameters etc.)
\end{itemize}

\section{Method}

The Method section (indicative length : 1 to 2 pages) should describe the mathematical aspects of the method in a summarized manner. Only the main steps that are useful for understanding should be highlighted. If relevant, some details on implementation can be provided (but only marginally).

\section{Data}
% \textit{The Data section (indicative length : 1 page) should provide a deep analysis of the data used for the experiment. In particular, we are interested here in your capacity to provide relevant and thoughtful feedbacks on the data and to demonstrate that you master some "data diagnosis" tools that have been dealt with in the lectures/tutorials.}

\subsection{BOLD5000 dataset}

We used the BOLD5000 database \cite{chang_bold5000_2019} in our experiments. It is a large-scale fMRI dataset that captures brain scans from four patients as they view over 5,000 images. The dataset covers a wide range of visual features, categories, and semantics, and can be used to test various hypotheses and models related to visual cognition. \\
fMRI produces 4D images, relying on the principle that localized neural activity induces changes in metabolism and blood flow.
% \red{, resulting in fluctuations in the concentrations of oxyhemoglobin\footnote{Oxyhaemoglobin are the red cells in the blood that carry oxygen} and deoxyhemoglobin\footnote{the same red cells after they have delivered the oxygen}. Oxyhaemoglobin and deoxyhaemoglobin have different magnetic properties\footnote{Oxyhemoglobin is diamagnetic, while deoxyhemoglobin is paramagnetic}, and they affect the local magnetic field in different ways.}
To record cerebral activity during functional sessions, the scanner is tuned to detect this "Blood Oxygen Level Dependent" (BOLD) signal. 
Brain activity is measured in sessions that span several minutes, during which participants are presented with a variety of images. Simultaneously, participants engage in a valence judgment task for each stimulus, expressing their preference using the descriptors "like", "neutral", and "dislike", which are encoded with labels 1, 2, and 3.\\
As the mapping between descriptors and labels was not provided, we inferred it ourselves.

\subsection{Time series extraction and diagnosis}

For each region of interests (ROIs) in the dataset, we extracted the time series, which represents the temporal activity in the brain. We used Nilearn -a Python library that provides tools for neuroimaging data analysis- to extract the signals.
Signal extraction is usually achieved by averaging the fMRI time series across the voxels in a region \cite{varoquaux_learning_2013}. 

In this study, time series were $z$-scored, \ie shifted to zero mean and scaled to unit variance. We also took into account the confounds in the extraction, as there were provided in the database. Confounds are variables that can affect the brain signal and are not of interest in the study. They can include head motion, physiological noise, and scanner artefacts. In order to extract clean signals from brain regions, it is important to remove the effects of confounds from the data. 
% This is done by regressing out the confounds from the data before extracting the signals. 

The time series obtained from the regions of interest (ROIs) do not appear to be noisy, as shown in Figure \ref{fig:plot_TS}. One can assume that signal are weak stationary as their mean and autocorrelation does not appear to vary over time. The spectrogram plotted in figure \ref{spectrogram_TS} shows that their spectral properties does noes not change over time. They all exhibit the same variability and mean as they were z-scored. Their values range between $-2.5$ and $2.5$, except for some regions that have extreme values or outliers (see Fig \ref{fig:boxplot_TS}). They seem to have the same spectral properties except for the retrosplenial complex (RSC) region in left and right hemisphere which has a more chaotic spectrum. Those results are shown in figure \ref{fig:spectral_TS}.

\subsection{Brain Graph Construction}

In order to represent the strength of axonal connection (structural connectivity), authors of \cite{huang_graph_2018} defined $A_{ij}$ as a simple count of the number of streamlines - \ie estimated individual fibers that connect the regions - using diffusion spectrum imaging. However, since we did not have access to this information, we defined the adjacency matrix $\A$ in another manner. In our setting, it corresponds to the connectivity matrix between brain regions (functional connectivity). Indeed, the extracted signals can be used to compute a correlation matrix between the regions \cite{varoquaux_learning_2013}. It is a common metric for computing the edges between the nodes, and is depicted in figure \ref{fig:connectivity}.






\section{Results}
The Result section (indicative length : 1 to 2 pages) should display numerical simulations on real data. If you re-used some existing implementations, it is expected that this section develops new experiments that were not present in the original article. Results should be discussed not only based on quantative scores but also on qualitative aspects. In particular (especially if your article focuses on black box methods), please provide some feedbacks whether the method was adapted to the data or not and whether the hypothesis behind the approach you used were validated or not.

\printbibliography
\clearpage
\appendix
\appendixpage % ajoute par defaut le titre "Appendices" au dessus de la première annexe


\end{document}
